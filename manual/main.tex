\documentclass[10pt]{report}
\usepackage{manual}

\title{
  {\huge Course Manual} \\
  \vspace{2cm}
  {\large \texttt{CAS CS 132}: \textit{Geometric Algorithms}} \\
  {\large Boston University}
}
\date{Spring 2026}

\begin{document}

\maketitle
\tableofcontents

\abstract{

  This is a manual for the course \texttt{CAS CS 132}:
  \textit{Geometric Algorithms}.  It contains a general overview of
  the course and its policies.  It does \textit{not} contain specifics
  about the material covered in the course; this appears on the
  \href{https://nmmull.github.io/cs132-s26/index.html}{course
    webpage}.  \Cref{fig:overview} contains an overview of the course.

  \begin{figure}
    \begin{tabular}{|l|l|}
      \hline
      Course Code & \texttt{CAS CS 132} \\
      Course Title & Geometric Algorithms \\
      Semester & Spring 2026 \\
      Instructor & Nathan Mull \\
      Teaching Fellow & Angelo Poulis \\
      Course Assistants & Gor Matcakian and Helen Zhou \\
      Meeting Time
      & Tuesday and Thursday, 11:00AM-12:15PM (A1) \\
      Meeting Location & \href{https://www.bu.edu/classrooms/classroom/cas-b12}{CAS B12} \\
      Midterm Date & March 5 (during lecture) \\
      Grade Breakdown
      & 15\% Assignments \\
      & 15\% Labs \\
      & 10\% Workshops \\
      & 20\% Quizzes \\
      & 20\% Midterm Exam \\
      & 20\% Final Exam \\
      \hline
    \end{tabular}
    \centering
    \caption{Course overview}
    \label{fig:overview}
  \end{figure}

}

\chapter{Week 0 To-Do list}

You should complete the following items within the first 48 hours of
the start of the semester.  Please reach out if you have concerns
about any of the items listed.

\begin{itemize}
\item
  [$\square$] Verify that you have access to a laptop computer during
  the semester
\item[$\square$] Verify that you know where the lecture is held
\item[$\square$] Verify that you know where the discussion section in
  which you're registered is held
\item[$\square$] Join Piazza with the following
  \href{https://piazza.com/bu/spring2026/cascs132}{sign-up link}
\item[$\square$] Join Gradescope with the following
  \href{https://www.gradescope.com/courses/1222810}{sign-up link}
  (Entry code: 6XXVDP)
\item[$\square$] Familiarize yourself with the
  \href{https://nmmull.github.io/cs132-s26/index.html}{course webpage}
\item[$\square$] \textit{(Optional)} Add Piazza, Gradescope, and
  course webpage as bookmarks in your Internet browser
\item[$\square$] Review the course calendar and determine which office
  hours you're able to attend
\item[$\square$] \textit{(Optional)} Add the course calendar to your
  own calendar
\item[$\square$] Read this manual in its entirety
\item[$\square$] Submit the assignment on Gradescope confirming that
  you've read this manual
\item[$\square$] If necessary, review basic programming concepts from
  \texttt{CAS CS 111}: \textit{Introduction to Computer Science} or an
  equivalent course
\item[$\square$] If necessary, review how to solve systems of linear
  equations (also called \textit{simultaneous equations}) in 2-4
  variables.
\end{itemize}

\chapter{General Information}

\texttt{CAS CS 132}: \textit{Geometric Algorithms} is an introduction
to \textbf{linear algebra} with a bent towards applications in
computer science.  One of its primary goals is to prepare students for
courses in the computer science curriculum at BU that require linear
algebra. Due to the fundamental nature of linear algebra\textemdash in
machine learning, in graph algorithms, in optimization, in
graphics\textemdash this list of courses is substantial; the following
are recently taught that have \texttt{CS132} as a prerequisite:
\begin{itemize}
\item \texttt{CAS CS 365}: \textit{Foundations of Data Science}
\item \texttt{CAS CS 440}: \textit{Introduction to Artificial Intelligence}
\item \texttt{CAS CS 480}: \textit{Introduction to Computer Graphics}
\item \texttt{CAS CS 506}: \textit{Data Science Tools and Applications}
\item \texttt{CAS CS 531}: \textit{Advanced Optimization Algorithms}
\item \texttt{CAS CS 541}: \textit{Applied Machine Learning}
\item \texttt{CAS CS 581}: \textit{Computational Fabrication}
\item \texttt{CAS CS 582}: \textit{Geometry Processing}
\item \texttt{CAS CS 585}: \textit{Image and Video Computing}
\end{itemize}
It should be made clear at the offset that this is \textit{not} a
course on computational geometry in the traditional sense; we will not
cover topics like triangulation, mesh generation, geometric data
structures, or convex programming.\footnote{In spite of the out-dated
description of the course in our registrar.} Instead, we'll be
covering topics within the scope of linear algebra proper, like matrix
factorization, eigenvectors and eigenvalues, Markov chains, linear
regression, and singular value decomposition (SVD). The
\enquote{geometric} part of the course title comes from the way we'll
learn to think geometrically, and how linear algebra helps us
recognize the utility of conceptualizing data
\textit{spatially}. Professor Mark Crovella has a note on this in the
\href{https://mcrovella.github.io/CS132-Geometric-Algorithms/landing_page.html}{preface
  of the text} we'll be using for this course, which we recommend
reading at the start of the semester.

This course is run as a hybrid math-and-computer-science course.
There are handwritten assignments in the style of a courses like
\texttt{CAS MA 123}: \textit{Calculus I} and programming assignments
in the style of courses like \texttt{CAS CS 111}: \textit{Introduction
  to Computer Science I}.  So you'll need to be comfortable both with
mathematical formalism and basic programming concepts.

Our primary goal is not to teach specific techniques and tools, but to
teach a way of reasoning about data and mathematical structures.
However, a deep dive into any mathematical subject requires some
amount of rote practice and memorization (of specific techniques and
tools).  Before continuing we'd simply like to emphasize that we're
passionate about this topic, and we hope that, as difficult as this
class may be at times, there's some (type 2) fun to be had.  We look
forward to learning with you all.

\section{Prerequisites}

The formal prerequisites for this course are:
\begin{itemize}
\item \texttt{CAS MA 123}: \textit{Calculus I}
\item \texttt{CAS CS 111}: \textit{Introduction to Computer Science}
\end{itemize}
What we are looking for in an incoming student of this course is basic
competence in programming and general mathematical maturity at the
level of a first course in calculus.\footnote{You don't need to
remember the details of calculus in order to be successful in this
course.}  We expect proficiency in the programming language
\href{https://www.python.org}{Python} at the level of a first course
in computer science.  If you have not used Python, then you'll have to
learn it within the first week of the course; we will \textit{not} be
covering basic programming concepts.

We also expect that you know how to solve systems of linear equations
in 2-4 variables; this topic is covered in most high school algebra
curricula, sometimes under the name \textit{simultaneous equations}.
We'll provide notes and references to review this material.

\section{Learning Objectives}

From this course we hope that you will:
\begin{itemize}
\item Learn the basic concepts of linear algebra, which can be applied
  to a wide range of fields including abstract algebra, mathematics
  modeling, numerical analysis, theoretical computer science, data
  science, machine learning, optimization, and graphics.
\item Internalize the language and terminology of linear algebra,
  particularly how to use this language to discuss concepts in the
  aforementioned fields of mathematics and computer science.
\item Practice using the programming tools which depend on linear
  algebraic concepts and are ubiquitous in the aforementioned fields.
\item Develop the intellectual toolkit required to take more advanced
  courses in the CS department at BU, and also to self-learn related
  concepts.  In particular, our primary goal is not to memorize
  (though this is also important) but to \textit{solve problems}.
\end{itemize}
\texttt{CS132} fulfills a single unit in BU Hub areas
\textbf{Quantitative Reasoning II} and \textbf{Digital/Multimedia
  Expression}. We believe the first distinction is obvious; any use of
mathematics to reason rigorously about real-world problems exercises
the kinds of skills expected of this Hub requirement.  To further
emphasize this point: we're not only interested in the mathematics of
linear algebra, but also in it's application to problems like
population dynamics, market prediction, machine learning, data
visualization, and generally a wide range of social and engineering
problems.

The second distinction requires justification: as we use linear
algebraic tools to analyze problems and systems, we'll also use
visualization tools to \textit{see} what our analysis tells us.  This
is why the \enquote{geometric} part of this course is so important: it
opens the possibility for us to use our spatial intuitions in
situations where these intuitions don't obviously apply.  Learn to
harness this skill is more art than science.

\section{Course Structure}

\subsection*{Lectures}

We hold lectures each week on Tuesday and Thursdays (see the registrar
and the course webpage for details).  During lecture, we cover
the material that is presented in the reading, do live coding
examples, and provide practice problems.  The material used in the
lecture is made available on the course webpage before the lecture
meeting.  Barring technical difficulties, recordings of the lecture
will be made available.

We won't take attendance during lecture (except during
worshops\textemdash see below for details) but it is highly recommended that you
attend, and we refer you to the BU
\href{https://www.bu.edu/academics/policies/attendance/}{Attendance}
policy.  You'll be expected to participate in lectures by working on
practice problems and occasionally discussing topics with the people
sitting around you.

\subsection*{Workshops}

Six lectures during the semesters will be run as workshops.  This
means that you will be expected to do some at-home learning
beforehand; we will spend a majority of the lecture time working on an
in-class activity in groups.  We will take attendance insofar as you
will have to submit the required material by the end of lecture.

\subsection*{Discussion Sections}

We hold discussion sections each week on Monday (see the registrar and
the course webpage for details).  The discussion sections have two
purposes.  On some weeks they will introduce the programming-based
labs (see below for details). On other weeks there will be proctored
quizzes (see below for details).  As with lectures, we won't take
attendance during discussion sections, but it is highly recommended
that you attend.

\section{Resources}

\subsection*{Material}

We'll primarily be using the
\href{https://mcrovella.github.io/CS132-Geometric-Algorithms/landing_page.html}{online
  text} written by Professor Mark Crovella for this course. The text
is a jupyter notebook which you can access online. All supplementary
material will be made available on the course webpage.

Much of the material in this online text is based on the textbook
\textit{Linear Algebra and its Applications} by David Lay, Steven Lay
and Judi McDonald.  We recommend taking a look at this textbook if you
need extra practice, particularly because of the number of exercises
in it.  We'll also occasionally refer to
\href{https://textbooks.math.gatech.edu/ila/}{Interactive Linear
  Algebra}, a beautiful text out of Georgia Tech by Dan Margalit and
Joseph Rabinoff.  Neither of these texts are formal requirements of
the course.

\subsection*{Programming}

The programming in this course is done in Python with the libraries:
NumPy, SciPy, NetworkX, scikit-learn, and matplotlib.  You're required
to set this up on you personal machine or on a machine that you'll
have access to throughout the semester.  You'll have an opportunity to
do this in your first discussion section.  Please attend office hours
and use Piazza if you need help troubleshooting. If you're worried
about access to technology, please contact us as soon as possible and
we can see what we can do, though we cannot make any guarantees.

\subsection*{Course Communication}

Course announcements and discussions will happen on Piazza. If you're
unfamiliar with Piazza, see their
\href{https://support.piazza.com/support/solutions/48000185443}{support
  page} for information and tutorials. Some policies regarding the use
of Piazza:
\begin{itemize}
\item
  \textit{Don't ask homework questions directly.} Formulate a question which
  will aid in your understanding, and will hopefully help others as
  well.
\item
  \textit{Don't give homework solutions directly.}
\item
  \textit{Piazza is as useful as it is active.} Teaching fellows and course
  assistants will be answering questions on Piazza, but don't hesitate
  to answer questions yourself.
\end{itemize}
Make sure to set notifications correctly so you can keep up with
updates regarding the course. \enquote{I didn't see the Piazza post
  about it} is never a valid excuse for missing a piece of
information.
\subsection*{Submission}

We'll be using Gradescope for assignment and lab submissions. If you
are unfamiliar with Gradescope, see their
\href{https://www.gradescope.com/get_started}{Get Started} page for
information and tutorials.

\chapter{Evaluation}

\begin{figure}
  \begin{tabular}{|l|l|}
    \hline
    15\% & Assignments (12 total, 2 dropped) \\
    15\% & Labs (6 total, 1 dropped) \\
    10\% & Workshops (6 total, 1 dropped) \\
    20\% & Quizzes (6 total, 1 dropped) \\
    20\% & Midterm Exam \\
    20\% & Final Exam \\
    \hline
  \end{tabular}
  \centering
  \caption{Grade breakdown}
  \label{fig:grades}
\end{figure}

The grading breakdown for this course is given in \Cref{fig:grades}.
The sites of evaluation are detailed below.  Your raw percentage will
be determined according to this breakdown and your final letter grade
is guaranteed to be at least what is determined by Wheelock College's
\href{https://www.bu.edu/academics/wheelock/policies/grades-course-credits-incomplete-coursework/}{Grading
  Scale}.\footnote{Formally we're part of the College of Arts and
Sciences (CAS), but this grading scale is standard.}  But, to borrow a
phrase from Professor Mark Bun: \enquote{to correct for the
  possibility of [quizzes] and exams being more difficult than
  anticipated, letter grades may be (significantly) increased above
  these guarantees.}  Specifically, we may retroactively curve exam
and quiz grades using a linear scale.\footnote{See
\href{https://divisbyzero.com/2008/12/22/how-to-curve-an-exam-and-assign-grades/}{this
  article} for details if you're interested.}

\section{Assignments}

Assignments are released weekly on Thursdays and are due a week later
on the following Thursday by 8:00PM.  See Gradescope and the calendar
on the course webpage for details.  Assignments consist of written
problems and are to be submitted as a pdf file via Gradescope.  There
are 12 assignments total.  We drop your lowest two assignment grade,
so only 10 assignments count towards your final grade in the course.
We don't accept late assignments under any circumstances.

You'll notice that, despite the fact that there are many assignments,
they account for a very small portion of your final grade.  This is,
in part, necessitated by the recent advances in LLMs which can easily
solve problems we ask on homework assignments, but it's also a
conscious decision we've made to ensure that students are engaging
with the material beyond what can be reasonably asked in a homework
assignment.  You should think of the assignments in this course as
\textit{accountability checks} in that they require to engage with the
material each week.  But the more effort you put into learning and
internalizing the material in the assignments the more successful
you'll be in the other evaluation sites of the course like the quizzes
and exams.\footnote{We understand that this approach to evaluation is
not universally liked, and that in-person evaluation can be
challenging.  We take this into consideration when we calibrate the
difficulty of the material and the workload we expect.}

\section{Labs}

Labs are released every other week on Mondays and are due a week and a
half later on the following Thursday by 8:00PM (so that the deadline
coincides with the assignment deadline).  See the calendar on the
course webpage for more details.  Labs consist of a programming task,
as well as a write-up on the results of the programming task.  There
are 6 labs total.  We drop your lowest lab grade, so only 5 labs count
towards your final grade in the course.  We don't accept late labs
under any circumstances.

The course staff will go over the basic concepts of the labs (and will
help you begin the lab) in the discussion sections which fall on lab
release dates.  We highly recommend you attend these sessions, as they
will provide useful context for completing the labs.

\section{Quizzes}

Quizzes occur every other week on Mondays during discussion sections.
They consist of 1-4 problems that are similar to the problems that
have appeared on previously submitted assignments.  If you've
completed the assignments and practiced the concepts therein, you're
expected to have no problem on the quizzes.  There are 6 quizzes
total.  We drop your lowest quiz grade, so only 5 quizzes count
towards your final grade in the course.

Quizzes are 30 minutes long, unless you have disability accommodations
which allow for more time, in which case you will be allowed more
time.  Due to the volume of quizzes and the size of the course we
cannot accommodate a separate time and space to take the quizzes.

\section{Midterm Exam}

The midterm exam is held during lecture.  It is be a written exam, and
is meant to verify that you are prepared to apply ideas from linear
algebra to real-world applications in the second half of the course.
You should expect the midterm exam to be more difficult than the
quizzes.

\section{Final Exam}

The date of the final exam will be determined later in the semester.
It will be a written exam and is meant to verify that you've
internalized the basic concepts of the course, and can also apply them
to solve novel problems.  You should expect the final exam to be the
same difficulty as the midterm exam.

\chapter{Policies}

There are a number of policies associated with this course, some
specific to the course and others which hold more generally in the
university.


\section{Diversity Statement}

Our aim is to present material in a way that respects the diversity of
the student body. If we fail to do this, please make us aware. Any
suggestions are welcome and appreciated. We also expect students to
appreciate and respect the unique opportunity they have to participate
in a diverse student body like ours.

\section{Disability Statement}

If you require disability accommodations, please contact us as soon as
possible. You should provide us with the appropriate documentation,
available through the \href{https://www.bu.edu/disability/}{Disability
  and Access Services}.  In order to receive accommodations, you
\textit{must} be in contact with us.

If there’s a policy that we're failing to comply with, please
reach out with suggestions. And if you’d like accommodations that
aren't covered by existing services or policies, feel free to contact
us and we can see what we can do.  We want everyone to feel able to
fully participate in the course.

\section{Sexual Misconduct}

Please read the
\href{https://www.bu.edu/policies/sexual-misconduct-title-ix-hr/}{Sexual
  Misconduct Policy} and review the entire page for information on
talking to someone confidentially about experiences of sexual
misconduct, filing a report, and any other relevant information. Above
all, you should feel safe, and able to be productive. If this is not
the case, please reach out to us or someone else immediately.

The members of the course staff are considered \enquote{mandated
  reporters} and are required to report cases of sexual misconduct.
Therefore, \textbf{we cannot guarantee the confidentiality of a
report}. We must provide our Title IX coordinator with relevant details
such as the names of those involved in the incident.  The university
will consider a request for confidentiality and respect it to the
extent possible.

With that in mind, if you come to any of us with questions or
concerns, we will handle the situation to the best of our ability and
connect you with available resources.


\section{Academic Integrity}

Please read the
\href{https://www.bu.edu/academics/policies/academic-conduct-code/}{Academic
  Conduct Code} and review the entire page for information about what
constitutes academic dishonesty and what penalties arise as a result
of violations of this code.  This is taken very seriously at BU and we
take it seriously in this courses.  There are a couple policies about
which we'll be strict:
\begin{itemize}
\item
  You must submit your own work for all assignments and labs.
  Submitting the same file as another student, or something notably
  similar (e.g., identical wording or code in large parts of the
  solution) is considered academic misconduct and will be handled
  accordingly.
\item
  Copying or information-sharing regarding in-class evaluations like
  quizzes and exams is considered academic misconduct and will be
  handled accordingly.
\end{itemize}
If you work with others, consult materials found on the Internet, or
use an AI assistant, you should cite your sources. This is a useful
skill in any setting, and so we recommend being as conservative as
possible regarding citations. In an assignment, these citations should
appear next to every corresponding problem (in comments if the
submission is code). Some examples:

\begin{itemize}
\item
  I discussed problem 1 and 2 with Leah Smith. She helped me
  understand X and Y aspects of the problem.
\item
  I saw the stack overflow post
  \texttt{stackoverflow.com/questions/6681284/python-numpy-arrays} which
  informed my solution.
\item
  I helped Zihan Guo with problem 4. I told them to try using X.
\item
  I asked ChatGPT “what’s the largest eigenvalue of this
matrix?”
\end{itemize}
When in doubt, err on the side of longer, more descriptive citations.
We do not consider missing or poor citations is a direct act of
academic misconduct, but we will consider this grounds for further
investigation in suspicious cases.  Above all, use your best judgment
and remember:
\begin{itemize}
\item
  We care about your success in this course. We provide a number of
  avenues to ask for help, please use them.
\item
  You will have to answer questions on quizzes and exams without
  external aids (and in interviews when you apply for a job).
\item
  If you don’t know how to start thinking about a problem, it’s okay
  to ask for pointers in office hours and on Piazza.
\item
  We have safeguards (like dropped homework assignments) in the case
  you are unable to complete an assignment. In other words, don't
  submit someone else's work when you can drop an assignment.
\end{itemize}

\section{Generative AI}

The problem of generative AI in higher education will likely occupy us
for the next decade or so.  The role of these tools in our lives is
still an open question, one with many possible answers.  But these
tools exists, and the university, for better or for worse, has made
them more accessible with the introduction of
\href{https://terriergpt.bu.edu/login}{TerrierGPT}, to which all
students of the university have access.  As such, all courses
(including ours) are changing their policies.  Keep in mind that this
is all an experiment.  We don't know if our policy makes sense in the
long term (or even now).  But it's our attempt to come to terms with
the appearance of these tools in our courses.

This semester, we've re-weighted evaluation sites in order to maintain
the policy that \textit{LLMs and AI assistances are allowed for use on
  assignments and labs}.  It has taken us some time to decide on this
policy, and we're still not completely sure that it'll work for us,
but it seems inevitable based on the current state of the field.

Of course, part of the reason we've introduced this policy is that
these tools are becoming ubiquitous.  It's not unlikely that you'll be
\textit{expected} to use AI assistants in your future job.  Likewise,
\textit{nearly all questions that we can ask you on assignments can be
  easily solved by existing models.}  This is why we have quizzes: it's
not enough to produce the work for assignments, you have to
demonstrate it in a closed-book setting.

An obligatory concluding remark: that existing models can solve our
assignments does not negate the value in knowing how to do them
without the help of external tools.  To draw an imperfect analogy, we
don't learn a new language in order to have memorized a vast
collection of words and grammar rules, but in order to
\textit{internalize} the language, and learn how to \textit{interact}
with it and in it.  This is our goal in this course and beyond.  AI
tools can be incredibly useful in the process of learning and
internalizing, but the internalizing is what we really want to
achieve, so we can look at a problem and \textit{sense} the underlying
structure to which we can apply our knowledge from this course.

\section{Additional Attendance Policies}

As we've noted, we won't take attendance in our course. Instead, we
remind you that, according to the
\href{https://www.bu.edu/academics/policies/attendance/}{Attendance}
policy at BU, you're required to attend the courses in which you're
registered.

\subsection*{Absence Due to Religious Observance}

According to the BU policy on
\href{https://www.bu.edu/academics/policies/absence-for-religious-reasons/}{Absence
  Due to Religious Observance}: you \enquote{shall be excused from any
  such examination or study or work requirement, and shall be provided
  with an opportunity to make up such examination, study, or work
  requirement that may have been missed because of such absence on any
  particular day; provided, however, that such makeup examination or
  work shall not create an unreasonable burden upon such school.}

\subsection*{Bereavement}

According to the BU policy on
\href{https://www.bu.edu/academics/policies/student-bereavement/}{Student
  Bereavement}: you \enquote{should be granted up to five weekdays of
  bereavement leave for the death of an immediate family member.} Your advisor should help you coordinate your leave.

\section{Additional Grading Policies}

\subsection*{Regrade Requests}

Regrade requests may be submitted on Gradescope for up to one week
after receiving the grade for an evaluation site.
Regrade requests will only be considered in the case that the grader
has made a mistake in grading.  Any regrade requests which solely
appeal for a higher grade will not be considered.

\subsection*{Grading Grievances}

According to the BU policy on
\href{https://www.bu.edu/academics/policies/policy-on-grade-grievances-for-undergraduate-students-in-boston-university-courses}{Grade
  Grievances}: you may \enquote{contest a final course grade received
  in a unit-bearing Boston University course when that grade is
  alleged by the student to be arbitrary.}  Read the policy for more
information.  We recommend contacting us before submitting a formal
appeal.

\subsection*{Incomplete Grades}

According to the BU policy on \href{
 https://www.bu.edu/academics/policies/incomplete-coursework/}{Incomplete
  Coursework}: \enquote{An incomplete grade (I) is used only when the
  student has conferred with the instructor prior to the submission of
  grades and offered acceptable reasons for the incomplete work. An
  incomplete grade may be appropriate when the student has
  participated in and completed requirements representing a majority
  of the course, and circumstances prevent the student from completing
  remaining requirements by the conclusion of the course.} In
particular, \textbf{you must contact us before the last day of the
  semester in order to receive an incomplete grade.}

\chapter{Closing Remarks}

Quite a bit goes into organizing a course as well as taking a course.
In light of my comments on citations, I'll note that much of what's in
this document is based on similar documents (often taken without
permission) by Mark Crovella, Mark Bun, Jonathan Appavoo, Preethi
Narayanan, Ravi Chugh, Andrew McNutt, and others we may be missing.
All told, we hope that most of this logistical information will be
overshadowed in your memory by the concepts of the course, and that we
can focus on having a good time doing math and programming.

\section{Course Agreement}

In addition to a manual, we also consider this document a contract. The following
is what you must agree to in order to remain in this course.

\begin{quote}
  \textit{By enrolling in this course, I am agreeing to the policies
    outlined in this document, and I will uphold them to the best of
    my ability. I will also, generally speaking, try to be a
    reasonable person and be nice and good and respectful to the
    people around me taking\textemdash and running \textemdash the
    course.  In return, I expect a high-quality learning experience
    and respect from those around me taking\textemdash and
    running\textemdash the course.}
\end{quote}

\section{University Resources}

There are quite a few BU resources, it can sometimes be overwhelming.
Here’s a small list of the ones we think are important.  If you’re
struggling in this course due to personal/health conditions, we can’t
guarantee we can help, but if you’re comfortable reaching out, feel
free to send us an email and we can see if we can point you towards
the correct resources. If you’re not comfortable reaching out to us,
that’s okay too, hopefully this list can help you find what you need.
Also, keep in mind you can post anonymously on Piazza if you want to
ask for help without including your name.
\begin{itemize}
\item \href{https://www.bu.edu/disability/}{Disability and Access Services}
\item \href{https://www.bu.edu/shs/}{Student Health Services}
\item \href{https://www.bu.edu/shs/outreach-prevention/}{Outreach and Prevention}
\item \href{https://www.bu.edu/shs/mental-health/}{Behavioral Medicine}
\item \href{https://www.bu.edu/shs/survivor-support/}{Survivor Support (SARP)}
\item \href{https://www.bu.edu/advising/educational-resource-center/}{Educational Resources Center}
\item \href{https://www.bu.edu/isso/}{International Students \& Scholars Office}
\end{itemize}
\end{document}
